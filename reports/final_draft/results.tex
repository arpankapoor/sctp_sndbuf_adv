\chapter{Results}

\section{Use Case Description}
The test bed was used to create a scenario in which there are multiple
senders and multiple receivers. Multiple flows were created and the
bottleneck between the two Raspberry Pis were fully utilized. The aim
was to classify the flows according to the bandwidth requirements. The
TBF qdisc did not ensure fair sharing of bandwidth. The SFQ qdisc gives
very good performance but fails for high number of flows \cite{lartc}. We
expect our method of classification can yield better results in reducing
average flow completion time in the above case.

\section{Test Results}
\begin{figure}[h]
  \centering
  \begin{gnuplot}[terminal=cairolatex]
    set xlabel 'Time (s)'
    set ylabel 'Percentage Send buffer Used\vspace{0.5cm}'
    set format y '%.0s'
    set key off
    set xrange [0:315]
    plot 'data/deep_tbf.dat' using ($1-5):2 lw 2 with lines,'data/prasad_tbf.dat' using ($1-5):2 lw 2 with lines
  \end{gnuplot}
  \caption{Percentage send buffer variation with 2 flows each having duration of
    300 seconds using the Token Bucket Filter qdisc with rate limited to
    1mbit/sec}
  \label{fig:tbf}
\end{figure}

\begin{figure}[h]
  \centering
  \begin{gnuplot}[terminal=cairolatex]
    set xlabel 'Time (s)'
    set ylabel 'Percentage Send buffer Used\vspace{0.5cm}'
    set format y '%.0s'
    set key off
    set xrange [0:310]
    plot 'data/deep_sfq.dat' using ($1-80): 2 lw 2 with lines,'data/prasad_sfq.dat' using ($1-80): 2 lw 2 with lines
  \end{gnuplot}
  \caption{Percentage send buffer variation with 2 flows each having duration of
    300 seconds using Stochastic Fair Queuing qdisc with rate limited to
    1mbit/sec}
  \label{fig:sfq}
\end{figure}

\clearpage

\section{Explanation of Test Results}
In figure \ref{fig:tbf}, one of the flows uses majority of the bandwidth for a
specific period of time during which the percentage send buffer occupancy of
the other host remains almost constant. On the whole, one of the end hosts
utilized four times the bandwidth utilized by the other end host.

In figure \ref{fig:sfq}, the bandwidth was shared equally between both the
flows. The percentage send buffer occupancy of both the hosts fluctuated
periodically during the test period.

SFQ is known to fail with a large number of flows. On testing with 4000 flows,
SFQ ensured that 94\% of the flows received equal bandwidth.
TODO: Testing with the modified kernel.
