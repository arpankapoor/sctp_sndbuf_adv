\chapter{Introduction}
Stream Control Transport Protocol (SCTP) is a reliable transport protocol
designed to transport Public Switched Telephone Network (PSTN) signaling
messages over IP networks, but is capable of broader applications.
Unlike TCP, SCTP offers sequenced delivery of user messages within multiple
unidirectional logical channels called streams.
Each SCTP endpoint is represented as a set of destination transport addresses,
one of which is the primary address. If the primary address becomes unreachable
SCTP chooses another destination transport address to route the messages
thereafter. This provides network-level fault tolerance and is called
multi-homing.
It also employs a security cookie mechanism during association initialization
to provide resistance to flooding and masquerade attacks.

Advertising the amount of backlogged data present in the sender's buffer can
help network operators evaluate the end-to-end performance of a connection
in a better way than that with the existing passive measurements.
This information can also be used to infer whether a connection is limited
by the network or the application.

\section{Problem Statement}
To propose a scheme to advertise send buffer occupancy information in SCTP,
implement it in the Linux kernel and study the performance of the same in
classifying the network flows in a congested network.

\section{Literature Survey}
RFC 3286 \cite{rfc3286} provides a high level introduction to the capabilities
supported by SCTP, while RFC 4960 \cite{rfc4960} describes the complete
protocol. Agache and Raiciu \cite{tcp-sndbufadv} propose a scheme to advertise
send buffer occupancy in TCP. \cite{budigerelinux} was used to study the state
machine employed in the Linux SCTP implementation. It was also used to
understand the SCTP packet flow within the kernel. \cite{lartc} provided with
an overview of the traffic control and routing mechanisms in the Linux kernel,
along with the userspace tools available for shaping and controlling the
traffic.
