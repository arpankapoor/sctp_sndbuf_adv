\documentclass[a4paper,11pt]{article}
\usepackage[english]{babel}
\usepackage[backend=biber]{biblatex}
\usepackage{csquotes}
\usepackage{geometry}
\usepackage[cleanup]{gnuplottex}
%\usepackage[top=1.2in, bottom=1in, left=1.4in, right=1.4in]{geometry}
\usepackage{hyperref}

\addbibresource{../../references.bib}

\hypersetup{
	hidelinks,
	pdftitle={SCTP Sendbuffer Advertising},
	pdfauthor={Arpan Kapoor, Deepak Sirone J, K Prasad Krishnan},
	pdfsubject={End Semester Report},
}


\title{SCTP Sendbuffer Advertising\\
	{\normalsize CS4089 Project\\
		End Semester Report}}

\author{Arpan Kapoor (B120555CS)\\
	Deepak Sirone J (B120097CS)\\
	K Prasad Krishnan (B120128CS)\\
	Guided By: Dr.~Vinod Pathari}

\begin{document}

\maketitle

\begin{abstract}
We propose to advertise sendbuffer occupancy in SCTP, i.e.\ 
the amount of backlogged data present in the sender's buffer.
\end{abstract}

\section{Introduction}
Stream Control Transport Protocol (SCTP) is a reliable transport protocol
designed to transport Public Switched Telephone Network (PSTN) signaling
messages over IP networks, but is capable of broader applications.
Unlike TCP, SCTP offers sequenced delivery of user messages within multiple
unidirectional logical channels called streams.
Each SCTP endpoint is represented as a set of destination transport addresses,
one of which is the primary address. If the primary address becomes unreachable
SCTP finds another destination transport address to route the messages 
thereafter. This provides network-level fault tolerance and is called
multi-homing.
It also employs a security cookie mechanism during association initialization
to provide resistance to flooding and masquerade attacks.

Advertising the amount of backlogged data present in the sender's buffer can
help network operators evaluate the end-to-end performance of a connection
in a better way than that with existing passive measurements.
This information can also be used to infer whether a connection is limited
by the network or by the application.

\section{Problem Statement}
To propose a scheme to advertise sendbuffer occupancy information,
implement it in the Linux kernel and study the performance and security
implications of the same.

\section{Literature Survey}
RFC 3286 \cite{rfc3286} provides a high level introduction to the capabilities
supported by SCTP, while RFC 4960 \cite{rfc4960} describes the complete
protocol. Agache and Raiciu \cite{tcp-sndbufadv} propose a scheme to advertise
sendbuffer occupancy in TCP. \cite{budigerelinux} was used to study the state
machine employed in the Linux SCTP implementation. It was also used to 
understand the SCTP packet flow within the kernel.

\section{Work Done}
Initially, we wrote a file transfer utility that uses SCTP as the
transport protocol.
We modified a kernel module called \texttt{sctp\_probe} to measure and plot
the sendbuffer size at regular intervals during a file transfer performed
using our userspace program.

\begin{figure}[h]
	\centering
	\begin{gnuplot}[terminal=cairolatex, terminaloptions={monochrome}]
		set xlabel 'Time (s)'
		set ylabel 'Sendbuffer Used (KB)\vspace{0.5cm}'
		set format y '%.0s'
		set key off
		set xrange [0:30]
		plot '../../code/sndbuff_test/data/c6.dat' every 500 with lines lw 2
	\end{gnuplot}
	\caption{Sendbuffer variation with random size packets.}
\end{figure}


We explored the Linux kernel SCTP implementation to understand how the
data from userspace is transformed into a SCTP packet and sent to the network
layer.
The data structures related to the state information, specifically
the out queue were studied in detail. The parameter corresponding to the
sendbuffer information, which is to be advertised was identified.


\subsection{Design}

\section{Future Work}
To design a working prototype of sendbuffer advertising for SCTP in the Linux
kernel and test it in a simulated network. Security implications of the
prototype will also be studied.

\printbibliography

\end{document}
