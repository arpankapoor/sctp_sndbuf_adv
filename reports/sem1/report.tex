\documentclass[a4paper,11pt]{article}
\usepackage[english]{babel}
\usepackage[backend=biber]{biblatex}
\usepackage{bytefield}
\usepackage{csquotes}
\usepackage{geometry}
\usepackage[cleanup]{gnuplottex}
\usepackage{hyperref}

\addbibresource{../../references.bib}

\hypersetup{
	hidelinks,
	pdftitle={SCTP Sendbuffer Advertising},
	pdfauthor={Arpan Kapoor, Deepak Sirone J, K Prasad Krishnan},
	pdfsubject={End Semester Report},
}


\title{SCTP Sendbuffer Advertising\\
	{\normalsize CS4089 Project\\
		End Semester Report}}

\author{Arpan Kapoor (B120555CS)\\
	Deepak Sirone J (B120097CS)\\
	K Prasad Krishnan (B120128CS)\\
	Guided By: Dr.~Vinod Pathari}

\begin{document}

\maketitle

\begin{abstract}
Advertising sendbuffer, i.e.
the amount of backlogged data present in the sender's buffer 
has been proposed for TCP and experiments show
improvements in application limited flows.
This project proposes to advertise the same in SCTP.
\end{abstract}

\section{Introduction}
Stream Control Transport Protocol (SCTP) is a reliable transport protocol
designed to transport Public Switched Telephone Network (PSTN) signaling
messages over IP networks, but is capable of broader applications.
Unlike TCP, SCTP offers sequenced delivery of user messages within multiple
unidirectional logical channels called streams.
Each SCTP endpoint is represented as a set of destination transport addresses,
one of which is the primary address. If the primary address becomes unreachable
SCTP chooses another destination transport address to route the messages 
thereafter. This provides network-level fault tolerance and is called
multi-homing.
It also employs a security cookie mechanism during association initialization
to provide resistance to flooding and masquerade attacks.

Advertising the amount of backlogged data present in the sender's buffer can
help network operators evaluate the end-to-end performance of a connection
in a better way than that with the existing passive measurements.
This information can also be used to infer whether a connection is limited
by the network or the application.

\section{Problem Statement}
To propose a scheme to advertise sendbuffer occupancy information,
implement it in the Linux kernel and study the performance and security
implications of the same.

\section{Literature Survey}
RFC 3286 \cite{rfc3286} provides a high level introduction to the capabilities
supported by SCTP, while RFC 4960 \cite{rfc4960} describes the complete
protocol. Agache and Raiciu \cite{tcp-sndbufadv} propose a scheme to advertise
sendbuffer occupancy in TCP. \cite{budigerelinux} was used to study the state
machine employed in the Linux SCTP implementation. It was also used to 
understand the SCTP packet flow within the kernel.

\section{Work Done}
Initially, we wrote a file transfer utility that uses SCTP as the
transport protocol.
We modified a kernel module called \texttt{sctp\_probe} to measure and plot
the sendbuffer size at regular intervals during a file transfer performed
using our userspace program.

\begin{figure}[h]
	\centering
	\begin{gnuplot}[terminal=cairolatex]
		set xlabel 'Time (s)'
		set ylabel 'Sendbuffer Used (KB)\vspace{0.5cm}'
		set format y '%.0s'
		set key off
		set xrange [0:3700]
		plot '../../code/file_transfer/datafile' using ($1-1447078873):2 every 500 with lines
	\end{gnuplot}
	\caption{Sendbuffer variation during a file transfer.}
\end{figure}

We explored the Linux kernel SCTP implementation to understand how the
data from userspace is transformed into a SCTP packet and sent to the network
layer.
The data structures related to the state information, specifically
the out queue were studied in detail. The parameter corresponding to the
sendbuffer information, which is to be advertised was identified.

\clearpage

\subsection{Prerequisite Terms}
\begin{itemize}
\item \textbf{SCTP Chunk} is a unit of information within an SCTP packet,
consisting of a chunk header and chunk-specific content.

\begin{figure}[h]
	\centering
	\begin{bytefield}[bitwidth=1.1em]{32}
	\bitheader{0-31}\\
	\bitbox{8}{Chunk Type} & \bitbox{8}{Chunk Flags} & \bitbox{16}{Chunk Length}\\
	\wordbox{3}{Chunk Value}
	\end{bytefield}
	\caption{SCTP Chunk Format}
\end{figure}

\item \textbf{SCTP Packet} consists of a common header followed by one or
	more chunks.
\begin{figure}[h]
	\centering
	\begin{bytefield}[bitwidth=1.1em]{32}
	\bitheader{0-31}\\
	\bitbox{16}{Source Port Number} & \bitbox{16}{Destination Port Number}\\
	\bitbox{32}{Verification Tag}\\
	\bitbox{32}{Checksum}\\
	\bitbox{32}{Chunk \#1}\\
	\bitbox{32}{\dots}\\
	\bitbox{32}{Chunk \#n}
	\end{bytefield}
	\caption{SCTP Packet Format}
\end{figure}

\item \textbf{SCTP Stream} referes to a sequence of user messages that are to be
	delivered to the upper layer protocol in order with respect to other
	messages within the same stream. This is unlike TCP, where it refers
	to a sequence of bytes.
\end{itemize}


\clearpage

\subsection{Design}
Each chunk is formatted with a Chunk Type field, a chunk-specific Flag field,
a Chunk Length field, and a Value field. Chunk Type, which takes a value
from 0 to 254, identifies the type of information contained in the Chunk
Value field. The Chunk Type values from 15 to 62, 64 to 126, 128 to 190 and
192 to 254 are not used presently in the SCTP implementation.
For advertising the sendbuffer occupancy,a new Chunk Type value is selected
from 128 to 190. RFC 4960 \cite{rfc4960} lays out the actions when
the processing endpoint does not recognize the Chunk Type (as is
the case here) depending on the highest order 2 bits. RFC 4960 \cite{rfc4960}
specifies that if the chunk is not recognised by the endpoint and the
highest order bits are 10, the chunk will be skipped without sending
an Unrecognized Chunk Type error chunk. This ensures that proposed
addition of a new Chunk Type would not affect hosts running the
unmodified SCTP linux implementation.

\begin{figure}[h]
	\centering
	\begin{bytefield}[bitwidth=1.1em]{32}
	\bitheader{0-31}\\
	\bitbox{8}{Chunk Type} & \bitbox{8}{Chunk Flags} & \bitbox{16}{Chunk Length}\\
	\bitbox{32}{Sendbuffer size}
	\end{bytefield}
	\caption{Proposed Chunk for sendbuffer advertisement}
\end{figure}

The overhead of 8 bytes which can come up in the proposed solution
is considerable as it is just 0.53\% of a 1500 byte packet.


\section{Future Work and Conclusions}
To design a working prototype of sendbuffer advertising for SCTP in the Linux
kernel and test it in a simulated network. Security implications of the
prototype will also be studied.


SDN (Software Defined Networking) based traffic engineering is an emerging area where
sendbuffer advertisement can be used for
performance optimization. Sendbuffer backlog data can be used by
an SDN enabled network switch to assign priorities to packets
depending on whether the flow is network limited or
application limited, to offer better quality of service to the flow.

The other direction is to use sendbuffer backlog for the identification
of congested links by SDN based traffic controllers. Controllers like
Hedera use the assumption that flows occupying a bandwidth higher than
a certain percentage of the host NIC's speed can be scheduled to
occupy an idle link with the assumption that the link will be filled
up. Experiments show that this does not always lead to an optimal
path allocation. The flow maybe application limited and may not
be capable of utilising the entire idle link bandwidth. Hence
determining if a flow is network limited or application limited
can be done accurately using sendbuffer advertisement.

\printbibliography

\end{document}
