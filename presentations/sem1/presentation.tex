\documentclass{beamer}
\usepackage[english]{babel}
\usepackage[backend=biber]{biblatex}
\usepackage{bytefield}
\usepackage{csquotes}
\setbeamertemplate{bibliography item}[text]

\addbibresource{./references.bib}

\title{SCTP Sendbuffer Advertising}
\subtitle{CS4089 Project\\
	End Semester Evaluation}

\author{Arpan Kapoor, Deepak Sirone J, K Prasad Krishnan\\
	Guided By:\\ Dr.~Vinod Pathari\\
	Mr.~V Anil Kumar, Principal Scientist, CSIR, Bengaluru}

\date{November 18, 2015}

\begin{document}

\begin{frame}
	\titlepage
\end{frame}

\begin{frame}{Outline}
	\tableofcontents
\end{frame}

\section{Introduction}
\begin{frame}{\insertsection}

\begin{itemize}
	\item Stream Control Transmission Protocol (SCTP):
	\begin{itemize}
		\item Supports multiple logical channels called streams
		\item Multi-homing
	\end{itemize}
\end{itemize}

\begin{itemize}
	\item Sendbuffer Advertising:
	\begin{itemize}
		\item each segment will carry the amount of backlogged data
			present in the sender's buffer.
	\end{itemize}
\end{itemize}

\end{frame}

\section{Problem Statement}
\begin{frame}{\insertsection}
\begin{itemize}
\item To propose a scheme to
\begin{itemize}
\item advertise sendbuffer occupancy information in SCTP
\item implement it in the Linux kernel and
\item study the performance and security implications of the same.
\end{itemize}
\end{itemize}
\end{frame}

\section{Prerequisite Terms}
\begin{frame}[fragile, allowframebreaks]
\frametitle{\insertsection}
\begin{itemize}

\item \textbf{SCTP Packet} consists of a common header followed by one or
	more chunks.

\begin{figure}
	\centering
	\begin{bytefield}{32}
	\bitheader{0-31}\\
	\bitbox{32}{Common Header}\\
	\bitbox{32}{Chunk \#1}\\
	\bitbox{32}{\dots}\\
	\bitbox{32}{Chunk \#n}
	\end{bytefield}
	\caption{SCTP Packet Format \cite{rfc4960}}
\end{figure}
\framebreak
\item \textbf{SCTP Chunk} is a unit of information within an SCTP packet,
consisting of a chunk header and chunk-specific content.

\begin{figure}
	\centering
	\begin{bytefield}{32}
	\bitheader{0-31}\\
	\bitbox{8}{Chunk Type} & \bitbox{8}{Chunk Flags} & \bitbox{16}{Chunk Length}\\
	\wordbox{2}{Chunk Value}
	\end{bytefield}
	\caption{SCTP Chunk Format \cite{rfc4960}}
\end{figure}
\item \textbf{Heartbeat Chunk} is used to probe the
reachability of a particular destination transport address.
\end{itemize}
\end{frame}

\section{Work Done}
\begin{frame}{\insertsection}
\begin{itemize}
\item Modified kernel module \texttt{sctp\_probe} to measure sendbuffer.
\item Explored Linux kernel SCTP implementation.
\item Identified parameter to be advertised.
\end{itemize}
\end{frame}

\section{Attempted Solution}
\begin{frame}{\insertsection}
\begin{itemize}
\item Encode the sendbuffer information as a variable length
	parameter in the Heartbeat chunk.
\item Problems:
	\begin{itemize}
		\item Can be disabled by upper layer.
		\item Is only sent to idle destination addresses.
	\end{itemize}
\end{itemize}
\end{frame}


\section{Design}
\begin{frame}[fragile]
\frametitle{\insertsection}
\begin{itemize}
\item New chunk type with Chunk Type value between 128 to 190.
\item Highest order 2 bits determine action to be taken if Chunk Type is
	unknown.
\item This ensures that unmodified hosts won't send a
	Unrecognized Chunk Type Error chunk upon reception.
\end{itemize}

\begin{figure}[h]
	\centering
	\begin{bytefield}{32}
	\bitheader{0-31}\\
	\bitbox{8}{Chunk Type} & \bitbox{8}{Chunk Flags} & \bitbox{16}{Chunk Length}\\
	\bitbox{32}{Sendbuffer size}
	\end{bytefield}
	\caption{Proposed Chunk for sendbuffer advertisement}
\end{figure}
\end{frame}

\section{Future Work}
\begin{frame}{\insertsection}
\begin{itemize}
\item Working prototype in Linux kernel.
\item To build a small testbed with few nodes and SDN routers.
\item Analyze the network performance using the testbed.
\end{itemize}
\end{frame}

\section{References}
\begin{frame}[allowframebreaks]
\frametitle<presentation>{\insertsection}
\nocite{*}
\printbibliography
\end{frame}

\end{document}
