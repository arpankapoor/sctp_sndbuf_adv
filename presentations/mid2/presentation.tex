\documentclass{beamer}
\usepackage[english]{babel}
\usepackage[backend=biber]{biblatex}
\usepackage{bytefield}
\usepackage{csquotes}
\setbeamertemplate{bibliography item}[text]

\addbibresource{./references.bib}

\title{SCTP Sendbuffer Advertising}
\subtitle{CS4089 Project\\
	End Semester Evaluation}

\author{Arpan Kapoor, Deepak Sirone J, K Prasad Krishnan\\
	Guided By:\\ Dr.~Vinod Pathari\\
	Mr.~V Anil Kumar, Principal Scientist, CSIR, Bengaluru}

\date{March 3, 2016}

\begin{document}

\begin{frame}
	\titlepage
\end{frame}

\begin{frame}{Outline}
	\tableofcontents
\end{frame}

\section{Introduction}
\begin{frame}{\insertsection}

\begin{itemize}
	\item Stream Control Transmission Protocol (SCTP):
	\begin{itemize}
		\item Supports multiple logical channels called streams
		\item Multi-homing
	\end{itemize}
\end{itemize}

\begin{itemize}
	\item Sendbuffer Advertising:
	\begin{itemize}
		\item each segment will carry the amount of backlogged data
			present in the sender's buffer.
	\end{itemize}
\end{itemize}

\end{frame}

\section{Problem Statement}
\begin{frame}{\insertsection}
\begin{itemize}
\item To propose a scheme to
\begin{itemize}
\item advertise sendbuffer occupancy information in SCTP
\item implement it in the Linux kernel and
\item study the performance and security implications of the same.
\end{itemize}
\end{itemize}
\end{frame}

\section{Work Done in the previous semester}
\begin{frame}{\insertsection}
\begin{itemize}
\item Modified kernel module \texttt{sctp\_probe} to measure sendbuffer.
\item Explored Linux kernel SCTP implementation.
\item Identified parameter to be advertised.
\end{itemize}
\end{frame}

\section{Design}
\begin{frame}[fragile]
\frametitle{\insertsection}
\begin{itemize}
\item New chunk type with Chunk Type value between 128 to 190.
\item Highest order 2 bits determine action to be taken if Chunk Type is
	unknown.
\item This ensures that unmodified hosts won't send a
	Unrecognized Chunk Type Error chunk upon reception.
\end{itemize}

\begin{figure}[h]
	\centering
	\begin{bytefield}{32}
	\bitheader{0-31}\\
	\bitbox{8}{Chunk Type} & \bitbox{8}{Chunk Flags} & \bitbox{16}{Chunk Length}\\
	\bitbox{32}{Sendbuffer size}
	\end{bytefield}
	\caption{Proposed Chunk for sendbuffer advertisement}
\end{figure}
\end{frame}

\section{Work Done in the current semester}
\begin{frame}{\insertsection}
\begin{itemize}
\item A working prototype of sendbuffer advertisement was implemented in the Linux kernel.
\begin{itemize}
	\item Added a timer with appropriate modifications to the state machine function table.
\end{itemize}
	\item The interval for sending the sendbuffer advertisement chunk can be modified at runtime.
\begin{itemize}
	\item Added a procfs entry to change the sendbuffer advertisement interval
\end{itemize}
\end{itemize}
\end{frame}

\section{Future Work}
\begin{frame}{\insertsection}
\begin{itemize}
\item To test the prototype of sendbuffer advertising for SCTP in the Linux kernel in a network.
\item There are several scheduling algorithms which prioritizes packets based on some
criteria. One of these priority based scheduling algorithms can be modified to consider
the sendbuffer information and to improve QoS for high volume flows.
\end{itemize}
\end{frame}

\section{References}
\begin{frame}[allowframebreaks]
\frametitle<presentation>{\insertsection}
\nocite{*}
\printbibliography
\end{frame}

\end{document}
